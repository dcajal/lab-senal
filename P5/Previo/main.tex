\documentclass[12pt]{article}
\usepackage[spanish]{babel}
\usepackage{float}
\usepackage[hidelinks]{hyperref}
\usepackage[utf8x]{inputenc}
\usepackage{hyperref}
\usepackage[style=listgroup,acronym,toc,hyperfirst,xindy]{glossaries}
\makeindex
\usepackage{graphicx}
\usepackage{wrapfig}
\graphicspath{{images/}}
\usepackage{parskip}
\usepackage{color}
\usepackage{listings}
\usepackage{subcaption}
\renewcommand{\lstlistingname}{Listado}
\usepackage{natbib}
\usepackage{url}
\usepackage{amsmath}
\makeglossaries
\usepackage[xindy]{imakeidx}
\usepackage{fancyhdr}
\usepackage{vmargin}
\usepackage{enumerate}
\usepackage[final]{pdfpages}

\lstset{
	basicstyle=\footnotesize,
	breakatwhitespace=false,         
	breaklines=true,                 
	captionpos=b,                    
	keepspaces=true,                
	numbers=left,                    
	numbersep=5pt,                  
	showspaces=false,                
	showstringspaces=false,
	showtabs=false,                  
	tabsize=4,
	frame=single
}

\lstset{language=Matlab,%
	%basicstyle=\color{red},
	breaklines=true,%
	morekeywords={matlab2tikz},
	keywordstyle=\color{blue},%
	morekeywords=[2]{1}, keywordstyle=[2]{\color{black}},
	identifierstyle=\color{black},%
	stringstyle=\color{lilas},
	commentstyle=\color{green},%
	showstringspaces=false,%without this there will be a symbol in the places where there is a space
	numbers=left,%
	numberstyle={\tiny \color{black}},% size of the numbers
	numbersep=9pt, % this defines how far the numbers are from the text
	emph=[1]{for,end,break},emphstyle=[1]\color{red}, %some words to emphasise
	%emph=[2]{word1,word2}, emphstyle=[2]{style},    
}

\usepackage{fancyhdr}
\usepackage{vmargin}
\setmarginsrb{3 cm}{2.5 cm}{3 cm}{2.5 cm}{1 cm}{1.5 cm}{1 cm}{1.5 cm}

\setglossarystyle{long}

% Acronym style
\newacronymstyle{ex-footnote}%
{%
	\GlsUseAcrEntryDispStyle{footnote}%
}%
{%
	\GlsUseAcrStyleDefs{footnote}%
	\renewcommand*{\genacrfullformat}[2]{%
		\firstacronymfont{\glsentryshort{##1}}##2%
		\expandafter\footnote\expandafter{\expandafter\glsentrylong\expandafter{##1}}%
	}%
	\renewcommand*{\Genacrfullformat}[2]{%
		\firstacronymfont{\Glsentryshort{##1}}##2%
		\expandafter\footnote\expandafter{\expandafter\glsentrylong\expandafter{##1}}%
	}%
	\renewcommand*{\genplacrfullformat}[2]{%
		\firstacronymfont{\glsentryshortpl{##1}}##2%
		\expandafter\footnote\expandafter{\expandafter\glsentrylongpl\expandafter{##1}}%
	}%
	\renewcommand*{\Genplacrfullformat}[2]{%
		\firstacronymfont{\Glsentryshortpl{##1}}##2%
		\expandafter\footnote\expandafter{\expandafter\glsentrylongpl\expandafter{##1}}%
	}%
}

\setacronymstyle{ex-footnote}



% Acronym list
\newacronym{oscar}{OSCAR}{Orbiting Satellite Carrying Amateur Radio}
\newacronym{satnogs}{SatNOGS}{Satellite Networked Operation Ground Stations}

% Title
\title{Estudio Previo Práctica 5}			
% Author
\author{Priscila Gómez Lizaga \\ Diego Cajal Orleans}								
% Date
\date{\today}										

\makeatletter
\let\thetitle\@title
\let\theauthor\@author
\let\thedate\@date
\makeatother

\pagestyle{fancy}
\fancyhf{}
\rhead{\theauthor}
\lhead{\thetitle}
\cfoot{\thepage}

\begin{document}

%%%%%%%%%%%%%%%%%%%%%%%%%%%%%%%%%%%%%%%%%%%%%%%%%%%%%%%%%%%%%%%%%%%%%%%%%%%%%%%%%%%%%%%%%

\begin{titlepage}
	\centering
    \vspace*{0.5 cm}
    \includegraphics[scale = 0.75]{eina.png}\\[1.0 cm]	% University Logo
%    \textsc{\LARGE University of Cape Town}\\[2.0 cm]	% University Name
	\textsc{\Large 30357}\\[0.5 cm]				% Course Code
	\textsc{\large laboratorio de señal y comunicaciones}\\[0.5 cm]				% Course Name
	\rule{\linewidth}{0.2 mm} \\[0.4 cm]
	\setlength{\baselineskip}{2\baselineskip}
	{ \huge \bfseries \thetitle}\\
	\rule{\linewidth}{0.2 mm} \\[1.5 cm]
	
	\begin{minipage}{0.4\textwidth}
		\begin{flushleft} \large
			\emph{Autores:}\\
			\theauthor
			\end{flushleft}
			\end{minipage}~
			\begin{minipage}{0.4\textwidth}
			\begin{flushright} \large
			\emph{Nia:} \\
			684061 \\
			620622
												% Your Student Number
		\end{flushright}
	\end{minipage}\\[2 cm]
	
	{\large \thedate}\\[2 cm]
 
	\vfill
	
\end{titlepage}

%%%%%%%%%%%%%%%%%%%%%%%%%%%%%%%%%%%%%%%%%%%%%%%%%%%%%%%%%%%%%%%%%%%%%%%%%%%%%%%%%%%%%%%%%

%\tableofcontents
\pagebreak

%%%%%%%%%%%%%%%%%%%%%%%%%%%%%%%%%%%%%%%%%%%%%%%%%%%%%%%%%%%%%%%%%%%%%%%%%%%%%%%%%%%%%%%%%

\section{Publicaciones de señales biomédicas}
A continuación se presentan algunos ejemplos de publicaciones del grupo de investigación en análisis de señales biomédicas de la Universidad de Zaragoza (BSICoS).\\

\textbf{Aplicación:} Estudio de la relación entre ritmo cardiaco y respiración (Respiratory Sinus Arrhythmia).\\
\textbf{Señales involucradas:} Variabilidad del ritmo cardiaco y respiración.\\
\textbf{Métodos de procesado:} Real Wavelet Biphase (Método original, propuesto en el mismo paper).\\
\textbf{Referencia:} Assessment of Quadratic Nonlinear
Cardiorespiratory Couplings During Tilt Table Test
by Means of Real Wavelet Biphase [2018]. Autores: Spyridon Kontaxis, Jesús Lázaro, Eduardo Gil, Pablo Laguna, Raquel Bailón.\\\\

\textbf{Aplicación:} Estudio del Sistema Nervioso Autónomo.\\
\textbf{Señales involucradas:} ECG, PPG (pulse-photoplethysmogra).\\
\textbf{Métodos de procesado:} Análisis espectral, filtros FIR, algoritmos basados en wavelet.\\
\textbf{Referencia:} Autonomic Nervous System Measurement in
Hyperbaric Environments using ECG and PPG signals [2018]. Autores: Alberto Hernando, María Dolores Peláez-Coca, María Teresa Lozano, Montserrat Aiger, David Izquierdo, Alberto
Sánchez, María Isabel López-Jurado, Ignacio Moura, Joaquín Fidalgo, Jesús Lázaro, Eduardo Gil.\\\\

\section{Filtro de fase cero}
La función \textit{filtfilt} de \textit{Matlab} filtra la señal de entrada normalmente y después voltea la salida y la vuelve a filtrar. Por último, voltea de nuevo la salida del segundo filtrado. Esto produce una función de transferencia total (después de ambos filtrados) con fase igual a cero, es decir, sin distorsión de fase. Para obtener la función de transferencia de forma analítica se hace uso de la siguiente propiedad de la transformada discreta de Fourier:

\begin{equation}
x[n]\leftrightarrow X(e^{j\omega}) \Leftrightarrow x[-n]\leftrightarrow X^*(e^{j\omega})
\end{equation}

Tras el primer filtrado, para una entrada $X(e^{j\omega})$, se tiene

$$Y(e^{j\omega}) = X(e^{j\omega})H(e^{j\omega})$$

al voltearlo, usando la propiedad (1) antes nombrada

$$Z(e^{j\omega}) = X^*(e^{j\omega})H^*(e^{j\omega})$$

En este punto se vuelve a filtrar con el mismo filtro

$$V(e^{j\omega}) = X^*(e^{j\omega})H^*(e^{j\omega})H(e^{j\omega}) = X^*(e^{j\omega})|H(e^{j\omega})|²$$

y se voltea de nuevo

$$W(e^{j\omega}) = X(e^{j\omega})|H(e^{j\omega})|²$$

El filtro resultante tiene una función de transferencia igual al cuadrado del valor absoluto de la función de transferencia original. Es \textbf{real} y positiva, por lo que no tiene distorsión de fase.

\section{Estimador de máxima verosimilitud}
Como se va a utilizar un estimador MLE para obtener los parámetros de amplitud e instante de comienzo: $[\hat{n}, \hat{a}] = argmax{ln(f(x|n_0,a))}$
En primer lugar, se obtiene $f(x|n_0,a)$ a partir de $x[n] = a\cdot s(n-n_0) + v(n)$. \\
\begin{gather*}
f(x|n_0,a) = \dfrac{1}{(2\pi\sigma^2)^{N/2}}\cdot exp\left[-\dfrac{1}{2\sigma^2} 
\left(\sum_{n=0}^{n_0-1}x^2(n) + \sum_{n=n_0}^{n_0+D-1}(x(n) - a\cdot(n-n_0))^2 +\sum_{n=n_0+D}^{N-1}x^2(n) \right)  \right]
\end{gather*}
A continuación se procede a realizar el logaritmo neperiano de la función $f(x|n_0,a)$
\begin{gather*}
ln(f(x|n_0,a)) = \left[ ln(1) - \frac{N}{2}ln(2\pi\sigma^2) \right] + \\
+\left[  -\dfrac{1}{2\sigma^2} 
\left(\sum_{n=0}^{n_0-1}x^2(n) + \sum_{n=n_0}^{n_0+D-1}(x(n) - a\cdot(n-n_0))^2 +\sum_{n=n_0+D}^{N-1}x^2(n) \right)  \right]  
\end{gather*}

Se desarrolla previamente la expresión $\sum_{n=n_0}^{n_0+D-1}(x(n) - a\cdot(n-n_0))^2$ para obtener $ln(f(x|n_0,a))$ de forma más simplificada.
\begin{gather*}
\sum_{n=n_0}^{n_0+D-1}(x(n) - a\cdot(n-n_0))^2 = \sum_{n=n_0}^{n_0+D-1} (x^2(n) + a^2 x^2(n-n_0) - a\cdot x(n)s(n-n_0)) 		
\end{gather*}

Sustituyendo la expresión anterior en $ln(f(x|n_0,a))$ se obtiene:
\begin{gather*}
ln(f(x|n_0,a)) = - \frac{N}{2}ln(2\pi\sigma^2) -\dfrac{1}{2\sigma^2} [ \sum_{n=0}^{n_0-1}x^2(n) + \\
+\sum_{n=n_0}^{n_0+D-1}x^2(n) + \sum_{n=n_0+D}^{N-1}x^2(n) + a^2\sum_{n=n_0}^{n_0+D-1}s^2(n-n_0) - 2a\sum_{n=n_0}^{n_0+D-1}x(n)s(n-n_0) ] =\\
=- \frac{N}{2}ln(2\pi\sigma^2) - \dfrac{1}{2\sigma ^2} \left[ \sum_{n=0}^{N-1}x^2(n) - \dfrac{a^2}{2\sigma^2}E_s + \dfrac{a}{\sigma^2}\sum_{n=n_0}^{n_0+D-1}x(n)s(n-n_0)\right] 
\end{gather*}

De los parámetros que debemos estimar, en primer lugar se obtiene la estimación de la amplitud aplicando $\dfrac{\partial ln(f(x|n_0,a))}{\partial a} = 0$
\begin{gather*}
ln(f(x|n_0,a)) = - \frac{N}{2}ln(2\pi\sigma^2) + \frac{1}{\sigma^2} \left( \dfrac{1}{2}\sum_{n=0}^{N-1}x^2(n) -\dfrac{a^2}{2}E_s
+a\sum_{n=n_0}^{n_0+D-1}x(n)s(n-n_0) \right) \\ 
\dfrac{\partial ln(f(x|n_0,a))}{\partial a} = \frac{1}{\sigma^2} \left( \dfrac{-2a}{2} E_s + \sum_{n=n_0}^{n_0+D-1}x(n)s(n-n_0) \right)\\
\frac{1}{\sigma^2} \left( \dfrac{-2a}{2} E_s + \sum_{n=n_0}^{n_0+D-1}x(n)s(n-n_0) \right) =0\\
\left( -a\cdot E_s + \sum_{n=n_0}^{n_0+D-1}x(n)s(n-n_0) \right) =0\\	
\sum_{n=n_0}^{n_0+D-1}x(n)s(n-n_0) = a\cdot E_s\\
a= \dfrac{\sum_{n=n_0}^{n_0+D-1}x(n)s(n-n_0)}{E_s} = \dfrac{y(n)}{E_s}
\end{gather*}
Finalmente, se obtiene el valor de la amplitud estimada. Si consideramos el sumatorio como la salida de un filtro adaptado al QRS patrón s(n), el instante en el que se maximiza es cuando $n=n_0$.

\begin{gather*}
\hat{a} = argmax{\dfrac{y(n)}{E_s}}=\dfrac{y(n_0)}{E_s}
\end{gather*}
\\

Como se trata de un problema de estimar ambas variables, una vez que se ha obtenido $\hat{a}$ se sustituye en $ln(f(x|n_0,a))$ para estimar $\hat{n}$. Hay que tener en cuenta en la dicha expresión solo el último término depende de $n_0$ por lo que habrá que maximizar $\hat{n_0}= argmax\{ a\sum_{n=n_0}^{n_0+D-1}x(n)s(n-n_0) \}$.
\begin{gather*}
\hat{n_0} = argmax\{ \hat{a}\sum_{n=n_0}^{n_0+D-1}x(n)s(n-n_0) \} = \\
=argmax\{ y(n_0) \cdot \sum_{n=n_0}^{n_0+D-1}x(n)s(n-n_0) \}
=argmax\{y(n_0) y(n)\}
\end{gather*}
Al igual que en el caso anterior, el instante en el que se maximiza ocurre cuando $n=n_0$ y de esta forma se obtiene $\hat{n}$.
\begin{gather*}
\hat{n_0} = argmax\{y(\hat{n_0})^2\}
\end{gather*}

\section{Validación del algoritmo de detección}
Se ha desarrollado la siguiente función para determinar la precisión del algoritmo de detección de complejos QRS.\\

\lstinputlisting[language=Matlab]{../Code/validate_QRS.m}

\section{Desalineamiento en el promediado}
El principal efecto del desalineamiento en el promediado es que puede haber un desalineamiento en el m-ciclo ya que cada ciclo debería aparecer en $t=mT$ pero en este caso podría aparecer en $t=mT+\theta$.

A continuación, se procede a obtener la expresión del espectro de la esperanza de dicho promedio.
\begin{gather*}
E\{ \hat{s} \} = \frac{1}{M} E\{ \sum_{i=1}^{M}s(n-\theta_i)  \}
\end{gather*}
Teniendo en cuenta que las variables aleatorias uniformes tienen la siguiente propiedad $E[x(n)] = \sum_{\theta = -\infty}^{\infty}x(n)p_{\theta}(\theta)$. Y, además, que la esperanza pasa a ser un valor constante, se obtiene que el sumatorio de M veces la esperanza es dicho valor multiplicado por M.
\begin{gather*}
E\{ \hat{s} \} = \frac{M}{M} \sum_{\theta = -\infty}^{\infty} s(n-\theta)p_{\theta}(\theta) = \sum_{\theta = -\infty}^{\infty} s(n-\theta)p_{\theta}(\theta)
\end{gather*}

Finalmente la expresión del espectro se consigue transformando al dominio frecuencial si consideramos que $E\{ \hat{s} \} = s(n)*p_{\theta}(\theta)$.
Transformando las señales por separado:
\begin{gather*}
s(n) \rightarrow S(e^{jw})\\
P_{\theta}(\theta) = \sum_{\theta = -\infty}^{\infty}p_{\theta}(\theta)e^{jw\theta}
\end{gather*}
La expresión final que se obtiene es la siguiente:
\begin{gather*}
E\{ \hat{S}(e^{jw}) \}=S(e^{jw})P_{\theta}(\theta)
\end{gather*}

\section{Algoritmo LMS}
A partir de los gráficos de la diapositiva 96 se deduce que los mejores valores para $\mu$ se obtiene cuando se acerca a cero. De los valores disponibles se elige $\mu = 0.01$.
El desajuste M se calcula de la siguiente forma:
\begin{gather*}
	M=\dfrac{\mu}{2}Tr[ R_{xx}] = \dfrac{\alpha}{3}
\end{gather*}
En nuestro caso, debido a que la matriz de autocorrelación de z[n] es la matriz identidad multiplicada por $1/L$ la traza que se obtiene es igual a la unidad.
\begin{gather*}
	M=\dfrac{\mu}{2}Tr[ R_{xx}] = =\frac{0.01}{2}\cdot 1 = 0.005
\end{gather*}
Para el cálculo del tiempo de convergencia se aplica
\begin{gather*}
	\tau_{mse,max} = \dfrac{3}{4\alpha}\dfrac{\sum_{k=0}^{L-1}\lambda_k}{\lambda_{min}}
\end{gather*}
En primer lugar es necesario conocer el valor de $\alpha$ a partir del desajuste M.
\begin{gather*}
	M=\dfrac{\alpha}{3} \to \alpha = 3M=3\cdot 0.005=0.015
\end{gather*}
También hay que tener en cuenta que el valor que toman los autovalores es $\lambda_k =1/L $.
\begin{gather*}
\tau_{mse,max} = \dfrac{3}{4\cdot 0.015}\dfrac{\sum_{k=0}^{L-1} 1/L}{1/L} = \dfrac{3}{4\cdot 0.015} \dfrac{L\cdot 1/L}{1/L}= 50\dfrac{1}{1/L} =50L
\end{gather*}
\end{document}
