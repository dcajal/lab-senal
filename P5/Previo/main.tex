\documentclass[12pt]{article}
\usepackage[spanish]{babel}
\usepackage{float}
\usepackage[hidelinks]{hyperref}
\usepackage[utf8x]{inputenc}
\usepackage{hyperref}
\usepackage[style=listgroup,acronym,toc,hyperfirst,xindy]{glossaries}
\makeindex
\usepackage{graphicx}
\usepackage{wrapfig}
\graphicspath{{images/}}
\usepackage{parskip}
\usepackage{color}
\usepackage{listings}
\usepackage{subcaption}
\renewcommand{\lstlistingname}{Listado}
\usepackage{natbib}
\usepackage{url}
\usepackage{amsmath}
\makeglossaries
\usepackage[xindy]{imakeidx}
\usepackage{fancyhdr}
\usepackage{vmargin}
\usepackage{enumerate}
\usepackage[final]{pdfpages}

\lstset{
	basicstyle=\footnotesize,
	breakatwhitespace=false,         
	breaklines=true,                 
	captionpos=b,                    
	keepspaces=true,                
	numbers=left,                    
	numbersep=5pt,                  
	showspaces=false,                
	showstringspaces=false,
	showtabs=false,                  
	tabsize=4,
	frame=single
}

\lstset{language=Matlab,%
	%basicstyle=\color{red},
	breaklines=true,%
	morekeywords={matlab2tikz},
	keywordstyle=\color{blue},%
	morekeywords=[2]{1}, keywordstyle=[2]{\color{black}},
	identifierstyle=\color{black},%
	stringstyle=\color{lilas},
	commentstyle=\color{green},%
	showstringspaces=false,%without this there will be a symbol in the places where there is a space
	numbers=left,%
	numberstyle={\tiny \color{black}},% size of the numbers
	numbersep=9pt, % this defines how far the numbers are from the text
	emph=[1]{for,end,break},emphstyle=[1]\color{red}, %some words to emphasise
	%emph=[2]{word1,word2}, emphstyle=[2]{style},    
}

\usepackage{fancyhdr}
\usepackage{vmargin}
\setmarginsrb{3 cm}{2.5 cm}{3 cm}{2.5 cm}{1 cm}{1.5 cm}{1 cm}{1.5 cm}

\setglossarystyle{long}

% Acronym style
\newacronymstyle{ex-footnote}%
{%
	\GlsUseAcrEntryDispStyle{footnote}%
}%
{%
	\GlsUseAcrStyleDefs{footnote}%
	\renewcommand*{\genacrfullformat}[2]{%
		\firstacronymfont{\glsentryshort{##1}}##2%
		\expandafter\footnote\expandafter{\expandafter\glsentrylong\expandafter{##1}}%
	}%
	\renewcommand*{\Genacrfullformat}[2]{%
		\firstacronymfont{\Glsentryshort{##1}}##2%
		\expandafter\footnote\expandafter{\expandafter\glsentrylong\expandafter{##1}}%
	}%
	\renewcommand*{\genplacrfullformat}[2]{%
		\firstacronymfont{\glsentryshortpl{##1}}##2%
		\expandafter\footnote\expandafter{\expandafter\glsentrylongpl\expandafter{##1}}%
	}%
	\renewcommand*{\Genplacrfullformat}[2]{%
		\firstacronymfont{\Glsentryshortpl{##1}}##2%
		\expandafter\footnote\expandafter{\expandafter\glsentrylongpl\expandafter{##1}}%
	}%
}

\setacronymstyle{ex-footnote}



% Acronym list
\newacronym{oscar}{OSCAR}{Orbiting Satellite Carrying Amateur Radio}
\newacronym{satnogs}{SatNOGS}{Satellite Networked Operation Ground Stations}

% Title
\title{Estudio Previo Práctica 5}			
% Author
\author{Priscila Gómez Lizaga \\ Diego Cajal Orleans}								
% Date
\date{\today}										

\makeatletter
\let\thetitle\@title
\let\theauthor\@author
\let\thedate\@date
\makeatother

\pagestyle{fancy}
\fancyhf{}
\rhead{\theauthor}
\lhead{\thetitle}
\cfoot{\thepage}

\begin{document}

%%%%%%%%%%%%%%%%%%%%%%%%%%%%%%%%%%%%%%%%%%%%%%%%%%%%%%%%%%%%%%%%%%%%%%%%%%%%%%%%%%%%%%%%%

\begin{titlepage}
	\centering
    \vspace*{0.5 cm}
    \includegraphics[scale = 0.75]{eina.png}\\[1.0 cm]	% University Logo
%    \textsc{\LARGE University of Cape Town}\\[2.0 cm]	% University Name
	\textsc{\Large 30357}\\[0.5 cm]				% Course Code
	\textsc{\large laboratorio de señal y comunicaciones}\\[0.5 cm]				% Course Name
	\rule{\linewidth}{0.2 mm} \\[0.4 cm]
	\setlength{\baselineskip}{2\baselineskip}
	{ \huge \bfseries \thetitle}\\
	\rule{\linewidth}{0.2 mm} \\[1.5 cm]
	
	\begin{minipage}{0.4\textwidth}
		\begin{flushleft} \large
			\emph{Autores:}\\
			\theauthor
			\end{flushleft}
			\end{minipage}~
			\begin{minipage}{0.4\textwidth}
			\begin{flushright} \large
			\emph{Nia:} \\
			684061 \\
			620622
												% Your Student Number
		\end{flushright}
	\end{minipage}\\[2 cm]
	
	{\large \thedate}\\[2 cm]
 
	\vfill
	
\end{titlepage}

%%%%%%%%%%%%%%%%%%%%%%%%%%%%%%%%%%%%%%%%%%%%%%%%%%%%%%%%%%%%%%%%%%%%%%%%%%%%%%%%%%%%%%%%%

%\tableofcontents
\pagebreak

%%%%%%%%%%%%%%%%%%%%%%%%%%%%%%%%%%%%%%%%%%%%%%%%%%%%%%%%%%%%%%%%%%%%%%%%%%%%%%%%%%%%%%%%%

\section{Publicaciones de señales biomédicas}
A continuación se presentan algunos ejemplos de publicaciones del grupo de investigación en análisis de señales biomédicas de la Universidad de Zaragoza (BSICoS).\\

\textbf{Aplicación:} Estudio de la relación entre ritmo cardiaco y respiración (Respiratory Sinus Arrhythmia).\\
\textbf{Señales involucradas:} Variabilidad del ritmo cardiaco y respiración.\\
\textbf{Métodos de procesado:} Real Wavelet Biphase (Método original, propuesto en el mismo paper).\\
\textbf{Referencia:} Assessment of Quadratic Nonlinear
Cardiorespiratory Couplings During Tilt Table Test
by Means of Real Wavelet Biphase [2018]. Autores: Spyridon Kontaxis, Jesús Lázaro, Eduardo Gil, Pablo Laguna, Raquel Bailón\\\\

\textbf{Aplicación:} Estudio del Sistema Nervioso Autónomo.\\
\textbf{Señales involucradas:} ECG, PPG (pulse-photoplethysmogra).\\
\textbf{Métodos de procesado:} Análisis espectral, filtros FIR, algoritmos basados en wavelet.\\
\textbf{Referencia:} Autonomic Nervous System Measurement in
Hyperbaric Environments using ECG and PPG signals [2018]. Autores: Alberto Hernando, María Dolores Peláez-Coca, María Teresa Lozano, Montserrat Aiger, David Izquierdo, Alberto
Sánchez, María Isabel López-Jurado, Ignacio Moura, Joaquín Fidalgo, Jesús Lázaro, Eduardo Gil.\\\\


%\lstinputlisting[language=Matlab]{../Code/distance.m}
%\lstinputlisting[language=Matlab]{../Code/distance_matrix.m}
\end{document}
