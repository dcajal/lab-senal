\documentclass[12pt]{article}
\usepackage[spanish]{babel}
\usepackage{float}
\usepackage[hidelinks]{hyperref}
\usepackage[utf8x]{inputenc}
\usepackage{hyperref}
\usepackage[style=listgroup,acronym,toc,hyperfirst,xindy]{glossaries}
\makeindex
\usepackage{graphicx}
\usepackage{wrapfig}
\graphicspath{{images/}}
\usepackage{parskip}
\usepackage{color}
\usepackage{listings}
\usepackage{subcaption}
\renewcommand{\lstlistingname}{Listado}
\usepackage{natbib}
\usepackage{url}
\usepackage{amsmath}
\makeglossaries
\usepackage[xindy]{imakeidx}
\usepackage{fancyhdr}
\usepackage{vmargin}
\usepackage{enumerate}
\usepackage[final]{pdfpages}

\lstset{
	basicstyle=\footnotesize,
	breakatwhitespace=false,         
	breaklines=true,                 
	captionpos=b,                    
	keepspaces=true,                
	numbers=left,                    
	numbersep=5pt,                  
	showspaces=false,                
	showstringspaces=false,
	showtabs=false,                  
	tabsize=4,
	frame=single
}

\lstset{language=Matlab,%
	%basicstyle=\color{red},
	breaklines=true,%
	morekeywords={matlab2tikz},
	keywordstyle=\color{blue},%
	morekeywords=[2]{1}, keywordstyle=[2]{\color{black}},
	identifierstyle=\color{black},%
	stringstyle=\color{lilas},
	commentstyle=\color{green},%
	showstringspaces=false,%without this there will be a symbol in the places where there is a space
	numbers=left,%
	numberstyle={\tiny \color{black}},% size of the numbers
	numbersep=9pt, % this defines how far the numbers are from the text
	emph=[1]{for,end,break},emphstyle=[1]\color{red}, %some words to emphasise
	%emph=[2]{word1,word2}, emphstyle=[2]{style},    
}

\usepackage{fancyhdr}
\usepackage{vmargin}
\setmarginsrb{3 cm}{2.5 cm}{3 cm}{2.5 cm}{1 cm}{1.5 cm}{1 cm}{1.5 cm}

\setglossarystyle{long}

% Acronym style
\newacronymstyle{ex-footnote}%
{%
	\GlsUseAcrEntryDispStyle{footnote}%
}%
{%
	\GlsUseAcrStyleDefs{footnote}%
	\renewcommand*{\genacrfullformat}[2]{%
		\firstacronymfont{\glsentryshort{##1}}##2%
		\expandafter\footnote\expandafter{\expandafter\glsentrylong\expandafter{##1}}%
	}%
	\renewcommand*{\Genacrfullformat}[2]{%
		\firstacronymfont{\Glsentryshort{##1}}##2%
		\expandafter\footnote\expandafter{\expandafter\glsentrylong\expandafter{##1}}%
	}%
	\renewcommand*{\genplacrfullformat}[2]{%
		\firstacronymfont{\glsentryshortpl{##1}}##2%
		\expandafter\footnote\expandafter{\expandafter\glsentrylongpl\expandafter{##1}}%
	}%
	\renewcommand*{\Genplacrfullformat}[2]{%
		\firstacronymfont{\Glsentryshortpl{##1}}##2%
		\expandafter\footnote\expandafter{\expandafter\glsentrylongpl\expandafter{##1}}%
	}%
}

\setacronymstyle{ex-footnote}



% Acronym list
\newacronym{oscar}{OSCAR}{Orbiting Satellite Carrying Amateur Radio}
\newacronym{satnogs}{SatNOGS}{Satellite Networked Operation Ground Stations}

% Title
\title{Estudio Previo Práctica 1}			
% Author
\author{Priscila Gómez Lizaga \\ Diego Cajal Orleans}								
% Date
\date{\today}										

\makeatletter
\let\thetitle\@title
\let\theauthor\@author
\let\thedate\@date
\makeatother

\pagestyle{fancy}
\fancyhf{}
\rhead{\theauthor}
\lhead{\thetitle}
\cfoot{\thepage}

\begin{document}

%%%%%%%%%%%%%%%%%%%%%%%%%%%%%%%%%%%%%%%%%%%%%%%%%%%%%%%%%%%%%%%%%%%%%%%%%%%%%%%%%%%%%%%%%

\begin{titlepage}
	\centering
    \vspace*{0.5 cm}
    \includegraphics[scale = 0.75]{eina.png}\\[1.0 cm]	% University Logo
%    \textsc{\LARGE University of Cape Town}\\[2.0 cm]	% University Name
	\textsc{\Large 30357}\\[0.5 cm]				% Course Code
	\textsc{\large laboratorio de señal y comunicaciones}\\[0.5 cm]				% Course Name
	\rule{\linewidth}{0.2 mm} \\[0.4 cm]
	\setlength{\baselineskip}{2\baselineskip}
	{ \huge \bfseries \thetitle}\\
	\rule{\linewidth}{0.2 mm} \\[1.5 cm]
	
	\begin{minipage}{0.4\textwidth}
		\begin{flushleft} \large
			\emph{Autores:}\\
			\theauthor
			\end{flushleft}
			\end{minipage}~
			\begin{minipage}{0.4\textwidth}
			\begin{flushright} \large
			\emph{Nia:} \\
			684061 \\
			620622
												% Your Student Number
		\end{flushright}
	\end{minipage}\\[2 cm]
	
	{\large \thedate}\\[2 cm]
 
	\vfill
	
\end{titlepage}

%%%%%%%%%%%%%%%%%%%%%%%%%%%%%%%%%%%%%%%%%%%%%%%%%%%%%%%%%%%%%%%%%%%%%%%%%%%%%%%%%%%%%%%%%

%\tableofcontents
\pagebreak

%%%%%%%%%%%%%%%%%%%%%%%%%%%%%%%%%%%%%%%%%%%%%%%%%%%%%%%%%%%%%%%%%%%%%%%%%%%%%%%%%%%%%%%%%

\section{Búsqueda bibliográfica DTW}
A continuación se presentan algunos ejemplos de aplicaciones reales del DTW.\\

\textbf{Aplicación:} Clasificador de insectos voladores\\
\textbf{Descripción:} El objetivo es obtener el espectro del sonido del insecto que permite determinar el tipo de insecto se trata. Prevención sanitaria.\\
\textbf{Referencia:} Dynamic Time Warping Averaging of Time Series allows Faster and more Accurate Classification [2014]. Autores: F. Petitjean G. Forestier G.I. Webb A.E. Nicholson Y. Chen E. Keog.\\\\

\textbf{Aplicación:} Indexación de series temporales\\
\textbf{Descripción:} Mejora de la medida de la distancia para series temporales con respecto a la distancia euclídea.\\
\textbf{Referencia:} Exact indexing of dynamic time warping [2005]. Autores: Eamonn Keogh, Chotirat Ann Ratanamahatana.\\\\

\textbf{Aplicación:} Creación de índices de búsqueda\\
\textbf{Descripción:} Automatización de indexación de textos históricos con ruido para facilitar búsquedas.\\
\textbf{Referencia:} Word image matching using dynamic time warping [2003]. Autores: T.M. Rath, R. Manmatha\\


\section{Distancia de Itakura-LPC}
Se tiene que demostrar que la energía del error de predicción de un segmento $x_0$ para un modelo de predicción lineal de $P$ coeficientes es:

$$
E_{x_0x_1} = a_1^TR_{x_ox_o}a_1
$$

Para ello partimos de la expresión del error de predicción
\begin{equation}
e[n] = x[n]-\sum_{i=1}^{P}a_ix[n-i]
\end{equation}

donde $ai$ son los coeficientes del filtro. Si modificamos el vector de coeficientes de forma que quede

$$
a_1=[1, -a_1, -a_2, ..., -a_P]
$$

podemos expresar el error de forma compacta

$$
e[n] = \sum_{i=0}^{P}a_{1i}x[n-i]
$$

que expresado en forma matricial queda

$$
e = a_{1}^TX
$$

La potencia del error de un LPC es la esperanza del error cuadrático
$$
e² = (a_{1}^TX)(a_{1}^TX)^T
$$
$$
e² = a_{1}^TXX^Ta_{1}
$$
\\
$$
E[e²] = E[a_{1}^TXX^Ta_{1}]
$$
$$
E[e²] = E_{x_0x_1} = a_{1}^TE[XX^T]a_{1}
$$
\\
$$
E_{x_0x_1} = a_{1}^TR_{x_0x_0}a_{1}
$$


\section{Respuesta frecuencial de un filtro IIR}
La respuesta frecuencial de un filtro IIR es $1/fft$ de los coeficientes del filtro. La transformada de Fourier hace un barrido sobre la circunferencia unidad en el espacio Z, obteniendo información de ceros. Como los coeficientes del filtro IIR están en el denominador (polos), se debe hacer el inverso multiplicativo.


\section{Búsqueda bibliográfica Mel}
De la misma manera que se ha hecho en el primer apartado, se presentan aplicaciones reales en las que se utiliza la escala perceptual auditiva Mel y/o los MFCC (Mel Frecuency Cepstral Coefficients).\\

\textbf{Aplicación:} Reconocimiento de voz\\
\textbf{Descripción:} Se utilizan los MFCCs como método de estracción de características y DTW para comparar patrones.\\
\textbf{Referencia:} Voice Recognition Algorithms using Mel Frequency Cepstral Coefficient (MFCC) and Dynamic Time Warping (DTW) Techniques [2010]. Autores: Lindasalwa Muda, Mumtaj Begam, I. Elamvazuthi\\\\

\textbf{Aplicación:} Detector automático de fases tempranas en el Parkinson.\\
\textbf{Descripción:} El objetivo es detectar la enfermedad de Parkinson a través del reconocimiento del habla, los análisis se llevan a cabo mediante parámetros a corto plazo y más precisamente mediante MFCC combinados con Modelos Mixtos Gaussianos. 
En el artículo se presenta el análisis durante cuatro tareas: vocales sostenidas, repeticiones de silabas rápidas, libertad de expresión y lectura, donde se adapta la metodología clásica.\\
\textbf{Referencia:} Automatic detection of early stages of Parkinson's disease through acoustic voice analysis with mel-frequency cepstral coefficients [2017]. Laetitia Jeancolas, Habib Benali, Badr-Eddine Benkelfat, Graziella Mangone, Jean-Christophe Corvol, Marie Vidailhet, Stephane Lehericy, Dijana Petrovska-Delacrétaz.5\\\\

\textbf{Aplicación:} Identificador de sarcasmo\\
\textbf{Descripción:} Debido a que cuando se habla de una forma sarcástica se produce un cambio en el tono, pitch... se propone en primer lugar extraer el audio de interés que pasará, a continuación, por un módulo que se encargará de reconocer el sarcasmo.\\
\textbf{Referencia:} Understanding sarcasm in speech using mel-frequency cepstral coefficient [2017]. Abhinav Mathur, Vikas Saxena, Sandeep K Singh\\\\

\textbf{Aplicación:} Clasificador de llantos infantiles\\
\textbf{Descripción:}  Reconocer el llanto de un niño a veces resulta complicado por lo que se ha desarrollado un sistema de clasificación utilizando MFCC y BNN (Backpropagation Neural Network). Se clasifican en 3 clases: hambriento, cansado o molesto.\\
\textbf{Referencia:} Infant’s Cry Sound Classification using Mel-Frequency Cepstrum Coefficients Feature Extraction and Backpropagation Neural Network [2016]. Yesy Diah Rosita y Hartarto Junaedi.\\


\section{Cálculo eficiente de distancias}
Código en \textit{Matlab} para la creación de una matriz de distancias entre dos vectores o matrices de forma eficiente.\\

\lstinputlisting[language=Matlab]{../Code/distance.m}
\lstinputlisting[language=Matlab]{../Code/distance_matrix.m}
\end{document}
